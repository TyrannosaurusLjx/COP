\documentclass[12pt, a4paper, oneside]{ctexart}
\usepackage{amsmath,extarrows, amsthm, amssymb, bm, graphicx, hyperref, geometry, mathrsfs,color}

\title{\huge\textbf{集合与实数集}}
\author{luojunxun}
\date{\today}
\linespread{2}%行间距
\geometry{left=2cm,right=2cm,top=2cm,bottom=2cm}%设置页面
\CTEXsetup[format={\Large\bfseries}]{section}%section左对齐

%定义环境
\newenvironment{Def}[1][def-name]{\par\noindent{\textit{(#1):}\small}}{\\\par}
\newenvironment{theorem}[1][Theorem-name]{\par\noindent \textbf{Theorem #1:}\textit}{\\\par}
\newenvironment{corollary}[1][corollary-name]{\par\noindent \textbf{Corollary #1:}\textit}{\\\par\vspace*{15pt}}
\newenvironment{lemma}[1][lemma-name]{\par\noindent \textbf{Lemma #1:}\textbf}{\\\par}
\renewenvironment{proof}{\par\noindent{\textit{Proof:}\small}}{\\\par}
\newenvironment{example}[1][example-name]{\par{\textbf{Example:}}}{\\\par}
\newenvironment{say}{\center{\textit{summary:}}}{\\\par}
\newenvironment{note}[1][note-name]{\par\textit{#1:}}{\\\par}
\newcommand{\qie}{\enspace\&\enspace}


\begin{document}
\maketitle

组合优化问题:从有限个可行解中找出最优可行解的优化问题称为组合优化问题

\section*{计算复杂性}
问题是带有若干参数的一个提问,给这些参数特定的值就得到一个实例,算法是求解该问题的通用步骤描述。
(这里的算法和数据结构中的算法不同,后者是程序执行的总步骤,而这里的算法是步骤的描述,可以通俗化)

规模(size):描述一个实例所需要的字节数,比如存储一个整数k所需要的字节数为$([log_2k]+1)+1$(标志结尾的空格)

最大数:实例的最大数就是在实例中出现过的最大整数

算法时间复杂性:算法的时间复杂性是一个关于实例规模x的函数f(x),他表示用该算法处理所有规模为x的实例中所需基本运算最多的那个实例的基本运算次数

如果函数f是一个多项式时间函数,称对应的算法是多项式时间算法,不能被次限制的称为指数时间算法

如果算法时间复杂性是关于实例规模x和最大数B的二元函数f(x,B)=O(P(x,b)),但算法不是多项式时间的,称为伪多项式时间算法,如O(nB);
这类算法的特点就在于实例规模比较小的时候计算速度很快,但是随着实例规模的增加,算法复杂度也会快速增加

有多项式算法的问题称为多项式时间可解问题类:记为$\mathcal{P}$

\section*{NP-完全性理论}

只回答是否的问题称为判定问题,判定问题和优化问题可以相互转化(NP-完全性理论研究判定问题(或者优化问题的判定形式))

非确定性算法多项式时间可解问题类:对一个问题,存在一个算法,使得对于任何一个回答为是的实例,该算法能猜出一个可行解,其算法规模,也就是描述可行解的字节数不超过实例的多项式函数,并且在不超过实例的多项式时间内能验证猜想正确,就称该问题为非确定性多项式时间可解问题类(nondeterministic polynomial solvable problem),记为$\mathcal{NP}$类

P问题是多项式时间可解问题,而NP问题是多项式时间可验证问题,自然有$P\subset NP$

NP类是非确定性图灵机多项式时间可解类

NP-C:NP完全问题:$p_1\in\mathcal{NP},\forall P_2\in\mathcal{NP},P_2\leq _pP_1\Rightarrow p_1\in\mathcal{NP}-C$

NP-C是NP中最难的问题,表现为如果某一个NP-C有多项式时间算法,那么每个NP问题都有多项式时间算法,且NP-C问题都是同等难度的,用归约证明即可

NP-C判定定理:$p\in\mathcal{NP},\exists p_0\in\mathcal{NP-}C,s.t.p\leq _pp_0\Rightarrow p\in\mathcal{NP}-C$



















% \begin{figure}[p]

%     \centerline{\includegraphics[width=1.2\linewidth,height=1.1\textheight]{name}}
%     \caption{课上习题}
%     \label{figure}

%\end{figure}



% \bibliographystyle{IEEEtran}
% \bibliography{reference}



\end{document}
